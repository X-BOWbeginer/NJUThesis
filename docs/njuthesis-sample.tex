%%%%%%%%%%%%%%%%%%%%%%%%%%%%%%%%%%%%%%%%%%%%%%%%%%%%%%%%%%%%%%%%%%%%%%
% njuthesis 示例模板 v1.4.1 2024-04-19
% https://github.com/nju-lug/NJUThesis
%
% 贡献者
% Yu XIONG @atxy-blip   Yichen ZHAO @FengChendian
% Song GAO @myandeg     Chang MA @glatavento
% Yilun SUN @HermitSun  Yinfeng LIN @linyinfeng
% Yukai Chou @Muzimuzhi
%
% 许可证
% LaTeX Project Public License(版本 1.3c 或更高)
%%%%%%%%%%%%%%%%%%%%%%%%%%%%%%%%%%%%%%%%%%%%%%%%%%%%%%%%%%%%%%%%%%%%%%

%---------------------------------------------------------------------
% 一些提升使用体验的小技巧:
%   1. 请务必使用 UTF-8 编码编写和保存本文档
%   2. 请务必使用 XeLaTeX 或 LuaLaTeX 引擎进行编译
%   3. 不保证接口稳定,写作前一定要留意版本号
%   4. 以百分号(%)开头的内容为注释,可以随意删除
%---------------------------------------------------------------------

%---------------------------------------------------------------------
% 请先阅读使用手册:
% http://mirrors.ctan.org/macros/unicodetex/latex/njuthesis/njuthesis.pdf
%---------------------------------------------------------------------

\documentclass[
    % 模板选项(注意右端逗号):
    %
    % type = bachelor|master|doctor|postdoc, % 文档类型,默认为本科生
    % degree = academic|professional,        % 学位类型,默认为学术型
    %
    % nl-cover,   % 是否需要国家图书馆封面,默认关闭
    % decl-page,  % 是否需要诚信承诺书或原创性声明,默认关闭
    %
    %   页面模式,详见手册说明
    % draft,                  % 开启草稿模式
    % anonymous,              % 开启盲审模式
    % minimal,                % 开启最小化模式
    %
    %   单双面模式,默认为适合印刷的双面模式
    % oneside,                % 单面模式,无空白页
    % twoside,                % 双面模式,每一章从奇数页开始
    %
    %   字体设置,不填写则自动调用系统预装字体,详见手册
    % fontset = win|mac|macoffice|fandol|none,
  ]{njuthesis}

% 模板选项设置,包括个人信息、外观样式等
% 较为冗长且一般不需要反复修改,我们把它放在单独的文件里
\input{njuthesis-setup.def}

% 自行载入所需宏包
\usepackage{subcaption} % 嵌套小幅图像,比 subfig 和 subfigure 更新更好
% \usepackage{siunitx} % 标准单位符号
% \usepackage{physics} % 物理百宝箱
% \usepackage[version=4]{mhchem} % 绘制分子式
\usepackage{listings} % 展示代码

\usepackage{xcolor}

% \usepackage{algorithm,algorithmic} % 展示算法伪代码

% 在导言区随意定制所需命令
% \DeclareMathOperator{\spn}{span}
% \NewDocumentCommand\mathbi{m}{\textbf{\em #1}}

% 开始编写论文
\begin{document}

%---------------------------------------------------------------------
%	封面、摘要、前言和目录
%---------------------------------------------------------------------

% 生成封面页
\maketitle

% 模板默认使用 \flushbottom,即底部平齐
% 效果更好,但可能出现 underfull \vbox 信息
% 以下命令用于抑制这些信息
\raggedbottom

\begin{abstract}

在系统的开发和维护过程中,错误定位工作作为排除系统错误的首要步骤,是一个关键的问题。随着软件规模和复杂性的不断增加,我们会面临着越来越多的错误定位和调试任务。
尽管软件系统在运行时会生成日志记录,但是如何有效地利用这些日志记录来定位和解决错误,发生错误时系统调用栈的状态仍然值得讨论。
传统朴素的错误定位方法往往需要大量的时间,并且在面对大型系统时尤为困难。因此,研究如何高效利用日志记录来定位和解决软件系统中的错误具有重要意义。本研究旨在探索并提出一种基于现有工具,更加准确的方法来利用日志记录进行软件错误定位,以帮助系统开发、维护,更快地定位错误,提高软件开发的效率。

在技术方面,选择使用clang生成待分析c++程序的的抽象语法分析树,然后借助libclang的python接口,完成对抽象语法分析树的解析,并生成相应的程序控制流图,以自动机的方式存储,接着需要将生成的非确定有限状态自动机,根据日志函数生成转化为确定有限状态自动机,这样便形成了基于代码中的日志关系生成日志链,供我们对运行日志进行推理,定位。同时,本程序使用了networkx与Gephi进行可视化展示。
\end{abstract}

\begin{abstract*}
  English abstract
\end{abstract*}

% 生成目录
\tableofcontents
% 生成图片清单
% \listoffigures
% 生成表格清单
% \listoftables

%---------------------------------------------------------------------
%	正文部分
%---------------------------------------------------------------------
\mainmatter

% 符号表
% 语法与 description 环境一致
% 两个可选参数依次为说明区域宽度、符号区域宽度
% 带星号的符号表(notation*)不会插入目录
% \begin{notation}[10cm]
%   \item[DFT] 密度泛函理论 (Density functional theory)
%   \item[DMRG] 密度矩阵重正化群 (Density-Matrix Reformation-Group)
% \end{notation}

% 建议将论文内容拆分为多个文件
% 即新建一个 chapters 文件夹
% 把每一章的内容单独放入一个 .tex 文件
% 然后在这里用 \include 导入,例如
%   \include{chapters/introduction}
%   \include{chapters/environments}

\chapter{绪论}


本章对大型系统开发中的日志分析需求进行了详细的探讨,
并提出了本文工作的目的和意义。
接着回顾了现有的相关研究,
分析了当前日志分析领域的发展现状。
几项关键研究文献展示了日志分析在系统运行信息的提取、
异常检测、日志记录位置的建议以及日志解析方法等方面的最新进展。
最后介绍了文章的研究内容和结构安排。
\section{设计目的及意义}
在大型系统的开发中,日志是分布在函数的不同分支中的,
开发人员在测试时需要基于系统的运行日志中的错误信息,来定位错误发生的位置,
其中包括了错误发生时的函数调用栈,比如路径敏感分析;
跨函数跨文件分析等上下文敏感分析。

传统分析方式耗时耗力,
需要开发人员对程序本身以及日志插入位置有着深刻的了解,
因此提供一种高效、准确的根据日志进行错误定位工具,
解决运行日志错误分析过程中的难点,并帮助用户快速定位和定界与特定日志相关的根因,
为程序开发和故障排查带来极大的便利和准确性。


\section{相关领域发展现状}
文献\cite{oliner2012advances}
介绍了日志分析的一些常见应用、被分析的日志类型以及分析方法,并阐明了其中的一些挑战。
日志分析是一个丰富的研究领域,它既重要又困难。
首先指出,日志的内容和格式因系统而异,甚至在系统内部的不同组件之间也可能存在差异。
日志的内容多种多样,包括了系统状态的片段信息,不同系统和组件的日志用途也各不相同。
例如,打印机驱动可能生成表示与打印机通信出现问题的消息,
而网络服务器可能记录哪些页面被请求以及何时请求。
当开发人员编写日志消息的打印语句时,它与程序源代码的上下文相关联。
然而,消息的内容经常排除了这个上下文。在没有了解到打印语句周围代码或了解程序执行路径的情况下,
可能会丢失部分消息的语义,也就是说,在没有上下文的情况下,日志消息可能很难理解。
机器学习技术,尤其是异常检测,常用于发现真正有用的日志消息。最近的研究通过分析源代码,
从传统文本日志中自动提取半结构化数据,并在从日志提取的特征上应用异常检测。
在几个开源系统和两个谷歌生产系统上,研究人员能够分析数十亿行日志,准确检测到人眼经常忽略的异常,
并将结果可视化为一页决策树图。
在统计异常检测方面仍然存在挑战。即使某些消息在统计意义上异常,
也可能没有进一步的证据表明这些消息是原因、症状还是简单无害。
此外,统计方法在很大程度上依赖于日志质量,特别是所记录的是否是“重要”事件。

文献\cite{li2020shall}
旨在解决开发人员在决定日志记录位置时面临的困难。
过少的日志可能导致缺少重要的系统执行信息,增加维护难度;
而过多的日志则可能掩盖真实问题并引发性能问题。
文献首先通过对七个开源系统进行综合手动研究,
揭示了日志记录位置的六个类别,
并发现开发人员通常在各种代码块中插入日志记录语句来记录执行信息。
基于观察到的模式,文献提出了一个深度学习框架,在代码块级别自动建议日志记录位置。
他们通过使用句法和语义信息对代码块级别的源代码进行建模。
实验结果表明,他们的模型在建议日志记录位置时平均达到80.1\%的平衡准确度;
跨系统的日志建议结果揭示了可能存在一个跨系统的隐含日志记录准则。
研究结果表明,他们能够准确提供更细粒度的日志记录位置建议,并且这些建议可能可以在不同系统间共享。


文献\cite{JSYJ2024031400B}
介绍了一种名为FMLogs的日志解析方法,该方法旨在将半结构化的原始日志解析为可阅读的日志模板。
与现有方法不同的是,FMLogs不仅注重对原始日志的解析,还着重考虑后期模板处理,
以提高解析精度。该方法利用正则表达式和阈值参数设计,以获得最佳性能。
此外,FMLogs采用字符级频率统计和MinHash方法来合并长度相同和不同的日志模板。
作者在7个真实数据集上进行了广泛的实验,结果显示,FMLogs的平均解析准确率为0.924,
F1-Score为0.983。实验结果表明,FMLogs是一种高效准确的日志解析方法,能够稳定地提供解析性能。



程序\cite{pyc-cfg}是一个完成度很高的Python控制流图构建器,
作为纯Python编写的控制流图生成器,它适用于几乎所有的ANSI C编程语言。
它通过构建来自Clang生成的抽象语法树来生成控制流图,并通过其与Libclang的Python绑定接口进行访问。
目前,该程序的作者正在改进代码以使其更符合Python的风格,并有可能进一步改进以处理复杂的C语言结构。

文献\cite{he2016experience}介绍了异常检测在现代大规模分布式系统管理中的重要作用。
日志作为记录系统运行信息的工具,在异常检测中被广泛应用。
传统上,开发人员或操作员通常通过关键字搜索和规则匹配手动检查日志。
然而,现代系统的规模和复杂性不断增加,导致日志数量急剧增加,使得手动检查的方法变得不可行。
为了减少人工工作量,提出了许多基于自动化日志分析的异常检测方法。
然而,开发人员可能仍然不知道应该采用哪种异常检测方法,因为这些方法之间缺乏综述和比较。
此外,即使开发人员决定采用一种异常检测方法,重新实现也需要很大的努力。
为了解决这些问题,作者提供了六种最先进的基于日志的异常检测方法的详细综述和评估,
包括三种有监督方法和三种无监督方法,并发布了一个开源工具包,方便复用。

故障诊断日志可以在软件系统出现故障时显著缩短系统恢复时间。
日志自动化工具可以帮助开发人员编写高质量的日志代码。
传统的日志自动化工具设计通过提取语法特征或总结代码模式来定义日志放置规则。
然而,这些方法存在局限性,因为日志放置远不止这些规则,
而是根据软件代码的意图来确定。
为了克服这些限制,文献6\cite{li2020swisslog}设计并实现了SmartLog,
这是一种基于意图感知的日志自动化工具。
为了描述日志语句的意图,该文作者提出了意图描述模型(IDM),
然后探索现有日志的意图,并从等效意图中挖掘日志规则,
并基于6个真实的开源项目进行了实验。
实验结果表明,SmartLog在日志放置准确性方面相比两种最先进的方法分别提高了43\%和16\%。
对于86个旨在添加日志的真实补丁,SmartLog覆盖了其中的57\%,而所有附加日志的开销不足1\%。

文献\cite{zou2014improving}提出了一个集成的故障日志分析平台(UiLog)
,用于收集和管理各种组件日志,为管理员快速定位故障和分析故障原因提供支持。
此系统已经部署在一个实际的云环境中,帮助管理员进行故障排查并找到故障的根本原因。
首先作者提出了一个新的故障日志分类方法,
使用故障关键词矩阵提高了准确性,减少了确定故障类型的时间和手动处理的工作量。
此外,作者改进了现有的日志关联分析方法。结合故障分类结果和时间窗口关联分析,
使用日志的故障类型作为确定时间窗口大小的一个因素,
提高了日志关联分析的准确性和故障根因定位的准确性。
最后,作者介绍了一个综合日志管理系统,帮助管理员快速掌握系统的运行状态,节省故障排查时间。


\section{本文研究内容}
本工具原型首先试图分析程序,得到描述日志记录信息的非确定有穷自动机,并将程序中的控制流信息转换为有穷自动机。该自动机通过状态和转换抽象地记录了程序可能的行为。然后使用自动机识别并分析日志序列,根据相应的转换序列来获得程序运行时刻的信息。

\section{本文结构安排}
本文的主要内容分成从以下六章进行论述。

第一章:绪论。介绍该工具原型设计的意义及目的,阐述了当前相关领域的发展情况,并大体概括了本工具原型的设计内容。

第二章:相关技术介绍,对本文中所应用的相关技术进行介绍描述,包括Clang,libeling库,抽象语法分析树,程序控制流图以及自动机等的相关知识。

第三章:工具原型设计,大致描述了整个工具原型的流程设计,对本工具原型中所采用主要数据结构进行了解释。

第四章:工具原型实现,提供了相应的开发环境安装流程,以及对工具原型每一个功能具体实现做出了详细的解释。

第五章:工具原型测试,针对工具原型功能的可行性验证,以及对工具原型稳定性的测试

第六章:总结与展望,对工具原型的整体功能做出总结,并讨论了工具原型未来的改进与拓展
\section{本章小结}
本章节介绍了现代软件系统中错误定位的重要性和挑战,并分析了现有的错误定位方法和工具,指出了它们在效率和精确度方面的不足,借此,本文提出了一种新的基于 Clang 和自动机的方法,以解析 C语言 程序的日志记录,并以此更有效地定位错误。


\chapter{系统技术介绍}

\section{clang}
clang是LLVM项目中提供的一个编译器前端,服务于整个C语言家族。相比于GCC,clang提供了更迅捷的编译速度,更好的静态诊断信息
以及更灵活的架构。

除了作为编译器以外,clang还可以作为一个库来使用,开发者不必下载整个程序,可以选择库中自己所需要的工具,只使用一部分的编译器的功能,
例如源代码的生成和分析,本系统便利用来clang来获取c++源文件的AST。
\section{libclang}
考虑到clang本身是用c++语言编写的,故所有API全是c++形式的,但是c++接口本身具有版本更新带来的不稳定性以及c++
语言特性产生的复杂性,LLVM官方在介绍中并不推荐普通开发者使用clang的c++接口。

官方更加推荐的是其提供的c语言接口,其具有操作简单,运行稳定的优点。但是c语言本身缺少高级的抽象,开发难度大,
本系统采用了基于这套c语言接口的python binding,它在使用上与c语言接口是基本一致,环境配置也比c++更简单,
唯一的缺点是无法像c++一样获取完整的源程序AST,并且相比c++也缺少了一部分更细节层面的API,所以在某些情况下,只能自己手动分析。
\section{自动机}
自动机是计算机科学中一种常见的抽象模型,它可以识别一串符合输入规范的序列,并判断接受或者拒绝它。通常情况下,自动机由一个状态合集,一个可接收的字符合集以及一组状态转移函数组成。
状态代表自动机在运行期间的某一种运行状态,不同的输入字符排列组合成自动机接收的序列,状态转移函数描述了状态之间的转移。自动机可以在不同的状态之间转换,并根据输入符号执行相应的动作。
\subsection{确定有限状态自动机}
确定有限状态自动机(DFA)是一种特殊类型的自动机。
在DFA中,对于给定的输入符号和当前状态,仅存在唯一的可能的下一个状态。
DFA 是一种状态转移图,其中每个状态都与一个输入符号相关联,并确定了进入下一个状态的路径。
它使用确定性转移函数来处理输入,并可以准确地确定接受或拒绝输入序列。
\subsection{非确定有限状态自动机}
非确定有限状态自动机(NFA)是另一种类型的自动机。
与DFA不同,NFA允许在给定的输入符号和状态下存在多个下一个状态。
这意味着在NFA中,给定一个输入符号和当前状态,可以有多个选择进行状态转移。
NFA使用非确定性转移函数来处理输入的序列,也存在空边,并且无法确定唯一的状态转移路径。

本系统中,我们根据生成的CFG首先生成的只是相应的NFA,但是要想准确定位错误,能够将某一个状态具体对应到代码的某一段,我们还需要将NFA转换成
DFA,对于这一转换,本系统采用的是子集构造法。
\subsection{子集构造法}
子集构造法(Subset Construction Method)是一种将非确定有限状态自动机(NFA)转换为等价的确定有限状态自动机(DFA)的算法。
该算法的基本思想是根据NFA的状态集合构造DFA的状态集合,并在状态转移函数中处理相应的转移。

以下是子集构造法的大体步骤:
\begin{enumerate}
	\item  初始状态:
 从NFA的初始状态开始,构造DFA的初始状态,即将初始状态作为DFA的初始状态。
    \item  状态转移:
对于每个DFA状态和每个输入符号,找到对应的NFA状态集合。
对于该输入符号,将NFA状态集合进行ε-闭包处理,即找到所有通过ε(空转移)可以到达的状态。
将得到的ε-闭包状态集合作为DFA状态的转移目标,并根据输入符号找到相应的转移。
    \item  状态集合构建:
对于每个新构造的DFA状态,重复第二步骤,直到没有新的状态可以构造。
    \item  判断接受状态:
根据NFA的接受状态集合来判断DFA的接受状态。
如果DFA的状态集合中包含任何一个NFA的接受状态,则将该DFA状态标记为接受状态。
\end{enumerate}
通过使用子集构造法,可以将NFA转换为具有确定性的DFA。


\section{抽象语法分析树}
抽象语法树(Abstract Syntax Tree,AST)是一个用于代表程序代码层次结构的树状数据结构。许多编译器,解释器以及静态分析工具都是以此为
开发基础。AST通过将代码语法拆解成了抽象的语法树结构,提供了一种以层次化,结构化理解和处理代码的方式。代码提示,格式化,debug工具,或者自动生成序列化代码等等,这些功能的开发都需要语法树里面的信息。
AST将代码中的各个元素(如表达式Expression,语句Statement,声明Declaration等)转化为语法树中的节点,而节点之间通过父子关系建立连接。
通常情况下,AST的根节点表示整个代码文件,然后每个语法结构(如函数、循环、条件语句等)都成为根节点的子节点。子节点可以继续有自己的子节点,以此类推,形成一个树状结构。

\begin{figure}[htbp]
	\centering
	\includegraphics[width=1\textwidth]{pictures/clang-AST.png}
	\caption{clang-AST}
	\label{fig:my_label}
\end{figure}


在clang生成的ast中,其根节点是TranslationUnitDecl,代表了一个c++源文件,我们可以从此开始遍历整个AST树,也可以获取已解析的的标识符表。
clang的AST节点并没有一个共同的“NODE”基类,大部分AST节点派生自 Type、 Decl、 DeclContext 或Stmt,还有一些节点属于自己的特定结构。
因此,要遍历完整的AST,需要从TranslationUnitDecl开始,然后递归地遍历从该节点可以到达的所有内容,并针对每种特定节点类型对这一信息进行编码。该算法被编码在一个RecursiveASTVisitor中。
值得庆幸的是,对于libclang而言,相应的api已经帮我们处理好了这些复杂的类型,我们只需要简单地把它们统一处理即可。


\section{程序控制流图}
程序控制流图程序控制流图(Program Control Flow Graph)是一种用于表示程序执行流程的图形结构。
它描述了程序中不同语句之间的控制流结构。
程序控制流图有助于可视化和分析程序的结构,并在软件工程中广泛应用于代码分析、特别是数据流分析相关的技术。

一个程序控制流图由以下几个基本元素组成:
\begin{itemize}
	\item 基本块(Basic Block):基本块是程序控制流图的基本单元,它是一组语句的顺序执行序列。基本块内部没有分支和跳转语句,只有一个进入点和一个退出点。基本块可以包含多行代码,但通常被简化为单个语句。
    \item 控制边(Control Edge):控制边用于表示程序控制的转移关系。它连接了控制流图中的不同基本块,指示程序的执行流从一个基本块转移到另一个基本块。控制边也可以表示条件分支、循环结构和其他控制转移。
    \item 进入点(Entry Point):进入点是控制流图中的起始点,它标识程序的入口,表示程序开始执行的位置。
    \item 退出点(Exit Point):退出点是控制流图中的终止点,它标识程序的出口,表示程序结束执行的位置。
\end{itemize}
值得注意的是,在本系统中,我们关注的主要是由日志函数,所以并不需要真正的按照基本块生成CFG,而是将一条语句作为程序控制流图中的基本块,这有助于简化CFG的整体结构。


\chapter{系统设计}

\section{数据结构设计}

对于每一条语句,我们都将生成一个自动机,它具有startNode和endNode两个state节点,并且
有一条边从startNode连到endNode,对于日志函数,边上的标签即是它的输出。
\begin{figure}[htbp]
	\centering
	\includegraphics[width=0.5\textwidth]{pictures/Cfg.png}
	\caption{自动机的数据结构}
	\label{fig:自动机的数据结构}
\end{figure}
整个系统生成CFG的过程就是递归生成自动机,并合并自动机的过程。
state节点在系统中用astNode存储,它保存了state节点的内置id。
\begin{figure}[htbp]
	\centering
	\includegraphics[width=0.5\textwidth]{pictures/fucTable.png}
	\caption{函数表}
	\label{fig:函数表}
\end{figure}
对于整个图结构,本系统采用的是networkx中的有向图来存储,同时我们也要存下自动机的字母表,用于对后续的DFA进行识别,另外,当遇到函数调用时,
为了知道入口函数的位置,我们也需要用一个函数表来通过函数的名字去查找它的入口和出口。
\section{程序流程设计}
对于本系统的主要模块,大致的流程是这样的:
\begin{enumerate}
	\item 使用clang扫描c++源文件,生成clang-AST
    \item 通过遍历AST,递归生成初始的CFG
    \item 对于CFG,获取日志函数的相关信息,生成相应的NFA
    \item 将NFA转换成为DFA
    \item 识别DFA
\end{enumerate}
\begin{figure}[htbp]
	\centering
	\includegraphics[width=1\textwidth]{pictures/系统流程.png}
	\caption{系统流程图}
	\label{fig:系统流程图}
\end{figure}


\chapter{系统实现}
\section{系统环境搭建}
对于windows系统,要运行本系统,首先需要安装python3,接着下载llvm和clang,然后适当的设置环境变量PYTHONPATH到clang.cindex,或者在py程序中的开头加上

\begin{lstlisting}[language=Python]
clang.cindex.Config.set_library_file(PATH to the libclang.dll)
\end{lstlisting}


然而,在笔者后续的探索中,发现在社区中有人已经替我们做好了这一切,
我们并不需要真的像上述那样在电脑上手动配置环境,如果想使用libclang来获取 C++ 语法树,只需要
\begin{lstlisting}[language=bash]
pip install libclang
\end{lstlisting}
不需要做其他任何操作。
此外,官方提供的包内缺少type hint,若想要比较舒适的开发环境,
可以在\href{https://github.com/16bit-ykiko/clang-related/blob/main/cindex.pyi}{此处}下载相应的提示文件并将它放在与cindex.py同一目录下,
即可获得代码提示。
\section{具体功能实现}
\subsection{从c++源程序到AST}
{\small 
\begin{lstlisting}[language=Python,caption=traverse]
def traverse(node: cx.Cursor, prefix="", is_last=True):
...
    # 分析函数体
    if node.kind is cx.CursorKind.FUNCTION_DECL:
        if node.is_definition():
...

    # 遍历子节点
    for child in children:
        traverse(child, new_prefix, child is children[-1])

\end{lstlisting}
}
该函数首先接受一个cindex.Cursor类型的node作为参数,根据相应的信息打印出AST,这相比于clang自带的AST更加清晰,包含了我们需要的信息,且便于调试。
接着便判断函数的类型是否为FUNCTION\_DECL,如果是,我们还需要判断它是否为函数定义,然后提取该函数定义的函数体,
递归构造自动机,同时维护函数表,将得到的函数名和入口出口保存。
最后,递归调用traverse函数本身。
{\small
\begin{lstlisting}[language=Python,caption=main]
def get_diag_info(diag):
...

def main():
...
\end{lstlisting}
}
首先我们创造了一个index对象,通常情况下,它就是多个翻译单元的集合,并且最终链接到一起。然后设置相应的参数来获取文件的翻译单元,即该AST的根节点,接着调用traverse,从
头开始遍历整个AST。此外,我们还打印了函数的诊断信息用于确认c++文件没有语法上的错误。
\begin{figure}[htbp]
	\centering
	\includegraphics[width=0.5\textwidth]{pictures/AST打印.png}
	\caption{AST打印}
	\label{fig:AST打印}
\end{figure}
\subsection{解析AST,生成CFG}
不同于c++,在python中语法树的基本节点都是由Cursor组成的,我们也是通过返回Cursor的
kind类型来判断节点对应的类型。下面对于Cursor中一些比较常用的属性做出解释:
\begin{itemize}
	\item spelling:节点的名称,若节点为函数就是函数名
	\item displayname:与前者差不多,但是对于函数会多一个参数
	\item type:节点的元素类型,比如int
	\item location:节点在文件中的位置,行数,列数等
\end{itemize}
 还有一些成员方法:
\begin{itemize}
	\item get\_children:获取所有的子节点,返回一个生成器
	\item get\_tokens:获取当前节点的所有tokens,也是返回一个生成器
    \item is\_definition:判断该节点属于函数定义还是声明
\end{itemize}
{\small
\begin{lstlisting}[language=Python,caption=Graph]
class Graph:
    def __init__(self):
...
    def draw(self):
...
    def toGexf(self):
...
\end{lstlisting}
}
对于基本的图结构,本系统设计了一个Graph类,采用了networkx中的有向图来存储,并对一些常用的方法进行封装,
这样在开发时,我们便只用关心对state节点的操作,它能自动的为节点分配id。其中draw,toGexf方法是用来debug和展示阶段性成果的。
{\small
\begin{lstlisting}[language=Python,caption=StateNode]
class AstNode:
    def __init__(self, id, cfg=None, isStart=False, isEnd=False, showInSubGraph=False):
...
\end{lstlisting}
}
对于一个state,本系统采用了一个AstNode类,它包含了id和所属的自动机,在创建时自动加入Graph中,注意一般一个自动机在创建时
会生成两个AstNode,代表了它的入口和出口state。
简单的说,AstNode就是代表了CFG中的state。
                   

对于一个自动机,我们会构造出一个Cfg类,下面是它的初始化函数:
{\small
\begin{lstlisting}[language=Python,caption=init]
def __init__(self, cursor: Cursor, childs=None, breakTarget=None, continueTarget=None, returnTarget=None,
             parent=None, isCondition=False, showInSubGraph=False, isEmpty=False, label=None):

    self.cursor = cursor

    # for while switch
    self.breakTarget = breakTarget
    self.continueTarget = continueTarget
...

    G.Cnode_num += 1
    self.gid = G.Cnode_num
    self.startNode = AstNode(id=str(self.gid) + "Start", cfg=self, isStart=True)
    self.endNode = AstNode(id=str(self.gid) + "End", cfg=self, isEnd=True)
 ...
    G.cfgs.append(self)
    self.buildChildCfg()
\end{lstlisting}
}
多数情况下,当我们想要创建一个自动机(Cfg类)时,只需要传入相应的cursor即可,对于节点内会出现break,continue的(如for,while),我们还需要
设置相应的跳转位置,接着init函数会自动为我们生成两个初始state,并调用buildChildCfg,生成该节点的子节点的自动机。

递归生成自动机的方法是整个系统中的重要一环,所以下面将对并调用buildChildCfg方法详细介绍。
\\
{\small
\begin{lstlisting}[language=Python]
cursorChilds = list(self.cursor.get_children())
\end{lstlisting}
}
首先我们通过相应的方法获取目标cursor的子节点的cursor,因为后续会涉及到对整体结构的判断,我们需要将返回的生成器实例化为list,便于操作。
\\
\subsubsection{break和continue}
{\small
\begin{lstlisting}[language=Python]
if len(cursorChilds) == 0:
    # BREAK,CONTINUE
...
\end{lstlisting}
}
对于没有子节点的情况,我们先做处理,在本系统中,出现这种情况只可能是break和continue,所以我们只需将自动机适当的连接到跳转target上即可。
\\
{\small
\begin{lstlisting}[language=Python]
elif self.cursor.kind == CursorKind.RETURN_STMT:
    ...
elif self.cursor.kind == CursorKind.DECL_STMT:
    ...
elif self.cursor.kind == CursorKind.BINARY_OPERATOR:
    ...
elif self.cursor.kind == CursorKind.UNARY_OPERATOR:
    ...
\end{lstlisting}
}
以上的节点也不是我们需要处理了,它们都是简单的线性结构,所以我们只是为其添加了标签,方便展示。

\subsubsection{函数调用}
{\small
\begin{lstlisting}[language=Python]
 elif self.cursor.kind == CursorKind.CALL_EXPR:
            if self.cursor.spelling == "log":
...
            else:
...
\end{lstlisting}
}
当节点是函数调用的时候,我们需要做的是先到Graph的函数表内,查找对应函数的入口和出口,如果是一般的函数,就直接把函数连接到调用的位置。
但是对于log日志函数,我们获取函数的参数,并将其添加到graph的输入表中,最后将参数添加到边上,并做特殊处理,方便后续生成NFA,DFA,毕竟最终生成的自动机的边上,
唯一有用的便是这些参数。

c语言家族有三种控制结构:顺序,选择和循环。
\subsubsection{顺序结构}
{\small
\begin{lstlisting}[language=Python]
elif self.cursor.kind == CursorKind.COMPOUND_STMT:
    G.removeEdge(self.startNode, self.endNode)
    self.showInSubGraph = True
    bottom = self.endNode
    for c in reversed(cursorChilds):  # 反向
...
\end{lstlisting}
}
普通的顺序结构反应到AST上就是复合语句节点,对于这类节点,
它的子节点需要按照顺序连接,注意到我们在连接的时候选择了从后往前,
这是因为当复合语句中存在跳转时,我们需要正确提供target,但是如果正常地从前往后生成,便可能
出现跳转目标还未生成的情况。所以这就是反着连的好处。
\begin{figure}[htbp]
	\centering
	\includegraphics[width=0.5\textwidth]{pictures/顺序结构.png}
	\caption{顺序结构}
	\label{fig:顺序结构}
\end{figure}

\subsubsection{循环结构}
循环结构中,我们主要处理for和while。
{\small
\begin{lstlisting}[language=Python]
elif self.cursor.kind == CursorKind.FOR_STMT:
    l = [False, False, False]
    index = 0
    # 双指针,解析for
    tokens = self.cursor.get_tokens()
    next_token = self.cursor.get_tokens()
    next(next_token)
    for i in tokens:
...
\end{lstlisting}
}
for循环的处理比较麻烦,主要的问题是,对于for语句,三个位置(init,condition,increasement)都有可能出现空的情况,然而libclang并没有提供相关的api供我们
判断,我们也无法从语法树上简单辨认出哪个是空的,而且空的位置也会导致整个for语句执行结构发生改变。

对于这类情况,我们采用的解决方案是获取节点的token,并用";"作为判空的依据,扫描for语句的第一行,同时返回一个长度为3的list,每个位置中用布尔值表示该位置是否为空。
至于空语句导致结构改变的问题,我们直接重新设计一种专门用于空语句的节点以应对这种情况,这样我们只需要按照正常情况连接整个for语句即可。
 \begin{figure}[htbp]
	\centering
	\includegraphics[width=0.5\textwidth]{pictures/for结构.png}
	\caption{for结构}
	\label{fig:for结构}
\end{figure}
{\small
\begin{lstlisting}[language=Python]
elif self.cursor.kind == CursorKind.WHILE_STMT:
    con = cursorChilds[0]
    stmts = cursorChilds[-1]
    conCfg = Cfg(con, parent=self, isCondition=True, showInSubGraph=True, label="WHILE_COND")

    # 提供breaktarget
    stmtsCfg = Cfg(stmts, breakTarget=self.endNode, continueTarget=self.startNode, parent=self,
                   showInSubGraph=True)
...
\end{lstlisting}
}
相反的,因为while语句条件必定不为空,我们只要按结构获取它的condition和body节点,依次连接即可。
 \begin{figure}[htbp]
	\centering
	\includegraphics[width=0.5\textwidth]{pictures/while结构.png}
	\caption{while结构}
	\label{fig:while结构}
\end{figure}
\subsubsection{选择结构}
{\small
\begin{lstlisting}[language=Python]
elif self.cursor.kind == CursorKind.IF_STMT:
    con = cursorChilds[0]
    then = cursorChilds[1]
    conCfg = Cfg(con, parent=self, isCondition=True, showInSubGraph=True, label="IF_COND")
    thenCfg = Cfg(then, parent=self, continueTarget=self.continueTarget, breakTarget=self.breakTarget)
   ...
    if len(cursorChilds) > 2:
       ...
    else:
       ...
\end{lstlisting}
}
if节点中,我们需要特殊考虑子节点没有else的情况,这时条件语句错误时将
直接跳到if节点的end语句。至于if,else if等等嵌套关系,递归调用自然会帮助我们正确处理。
 \begin{figure}[htbp]
	\centering
	\includegraphics[width=0.3\textwidth]{pictures/if结构.png}
	\caption{if结构}
	\label{fig:if结构}
\end{figure}
{\small
\begin{lstlisting}[language=Python]
elif self.cursor.kind == CursorKind.SWITCH_STMT:
    cases = cursorChilds[-1]
    if cases.kind == CursorKind.COMPOUND_STMT:
        casesCfg = Cfg(cases, parent=self, breakTarget=self.endNode)
        ...

elif self.cursor.kind == CursorKind.CASE_STMT:
    con = cursorChilds[0]
    then = cursorChilds[1]
    conCfg = Cfg(con, parent=self, isCondition=True, showInSubGraph=True, label="CASE_COND")
    thenCfg = Cfg(then, parent=self, continueTarget=self.continueTarget, breakTarget=self.breakTarget)
    ...

elif self.cursor.kind == CursorKind.DEFAULT_STMT:
    ...
\end{lstlisting}
}
选择结构中还有一部分是switch语句节点,在语法树上,多个case被包在一个复合语句中,
但是我们并不能直接在上层处理所有的case,因为不能保证复合语句只有case节点。
所以正确的做法是把问题留给下层,单独处理每一个case。

对于单个case,它们都有类似if一样的condition和body,同样的,当条件不匹配时,
我们将直接跳转到下一条语句,但是如果成功匹配,运行完case的body后我们将
退出整个switch语句。
 \begin{figure}[htbp]
	\centering
	\includegraphics[width=0.5\textwidth]{pictures/switch结构.png}
	\caption{switch结构}
	\label{fig:switch结构}
\end{figure}
\subsection{从CFG生成NFA,最终生成DFA}

\chapter{系统测试}
\section{测试目的}
\section{测试内容}
\section{测试方法}






%---------------------------------------------------------------------
%	参考文献
%---------------------------------------------------------------------

% 生成参考文献页
\printbibliography

%---------------------------------------------------------------------
%	致谢
%---------------------------------------------------------------------

\begin{acknowledgement}

\end{acknowledgement}

%---------------------------------------------------------------------
%	学术简历
%---------------------------------------------------------------------

% 详见手册中“成果列表”一节
% \njuchapter{学术成果}
% \njupaperlist[攻读博士学位期间发表的学术论文]{preskill2018}

%---------------------------------------------------------------------
%	附录部分
%---------------------------------------------------------------------

% 附录部分使用单独的字母序号
%\appendix

% 可以在这里插入补充材料
%\chapter{正文中涉及的数据及源代码}
%\dots

% 完工
\end{document}
