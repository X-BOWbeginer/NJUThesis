
\chapter{绪论}


本章对大型系统开发中的日志分析需求进行了详细的探讨,
并提出了本文工作的目的和意义。
接着回顾了现有的相关研究,
分析了当前日志分析领域的发展现状。
几项关键研究文献展示了日志分析在系统运行信息的提取、
异常检测、日志记录位置的建议以及日志解析方法等方面的最新进展。
最后介绍了文章的研究内容和结构安排。
\section{设计目的及意义}
在大型系统的开发中,日志是分布在函数的不同分支中的,
开发人员在测试时需要基于系统的运行日志中的错误信息,来定位错误发生的位置,
其中包括了错误发生时的函数调用栈,比如路径敏感分析;
跨函数跨文件分析等上下文敏感分析。

传统分析方式耗时耗力,
需要开发人员对程序本身以及日志插入位置有着深刻的了解,
因此提供一种高效、准确的根据日志进行错误定位工具,
解决运行日志错误分析过程中的难点,并帮助用户快速定位和定界与特定日志相关的根因,
为程序开发和故障排查带来极大的便利和准确性。


\section{相关领域发展现状}
文献\cite{oliner2012advances}
介绍了日志分析的一些常见应用、被分析的日志类型以及分析方法,并阐明了其中的一些挑战。
日志分析是一个丰富的研究领域,它既重要又困难。
首先指出,日志的内容和格式因系统而异,甚至在系统内部的不同组件之间也可能存在差异。
日志的内容多种多样,包括了系统状态的片段信息,不同系统和组件的日志用途也各不相同。
例如,打印机驱动可能生成表示与打印机通信出现问题的消息,
而网络服务器可能记录哪些页面被请求以及何时请求。
当开发人员编写日志消息的打印语句时,它与程序源代码的上下文相关联。
然而,消息的内容经常排除了这个上下文。在没有了解到打印语句周围代码或了解程序执行路径的情况下,
可能会丢失部分消息的语义,也就是说,在没有上下文的情况下,日志消息可能很难理解。
机器学习技术,尤其是异常检测,常用于发现真正有用的日志消息。最近的研究通过分析源代码,
从传统文本日志中自动提取半结构化数据,并在从日志提取的特征上应用异常检测。
在几个开源系统和两个谷歌生产系统上,研究人员能够分析数十亿行日志,准确检测到人眼经常忽略的异常,
并将结果可视化为一页决策树图。
在统计异常检测方面仍然存在挑战。即使某些消息在统计意义上异常,
也可能没有进一步的证据表明这些消息是原因、症状还是简单无害。
此外,统计方法在很大程度上依赖于日志质量,特别是所记录的是否是“重要”事件。

文献\cite{li2020shall}
旨在解决开发人员在决定日志记录位置时面临的困难。
过少的日志可能导致缺少重要的系统执行信息,增加维护难度;
而过多的日志则可能掩盖真实问题并引发性能问题。
文献首先通过对七个开源系统进行综合手动研究,
揭示了日志记录位置的六个类别,
并发现开发人员通常在各种代码块中插入日志记录语句来记录执行信息。
基于观察到的模式,文献提出了一个深度学习框架,在代码块级别自动建议日志记录位置。
他们通过使用句法和语义信息对代码块级别的源代码进行建模。
实验结果表明,他们的模型在建议日志记录位置时平均达到80.1\%的平衡准确度;
跨系统的日志建议结果揭示了可能存在一个跨系统的隐含日志记录准则。
研究结果表明,他们能够准确提供更细粒度的日志记录位置建议,并且这些建议可能可以在不同系统间共享。


文献\cite{JSYJ2024031400B}
介绍了一种名为FMLogs的日志解析方法,该方法旨在将半结构化的原始日志解析为可阅读的日志模板。
与现有方法不同的是,FMLogs不仅注重对原始日志的解析,还着重考虑后期模板处理,
以提高解析精度。该方法利用正则表达式和阈值参数设计,以获得最佳性能。
此外,FMLogs采用字符级频率统计和MinHash方法来合并长度相同和不同的日志模板。
作者在7个真实数据集上进行了广泛的实验,结果显示,FMLogs的平均解析准确率为0.924,
F1-Score为0.983。实验结果表明,FMLogs是一种高效准确的日志解析方法,能够稳定地提供解析性能。



程序\cite{pyc-cfg}是一个完成度很高的Python控制流图构建器,
作为纯Python编写的控制流图生成器,它适用于几乎所有的ANSI C编程语言。
它通过构建来自Clang生成的抽象语法树来生成控制流图,并通过其与Libclang的Python绑定接口进行访问。
目前,该程序的作者正在改进代码以使其更符合Python的风格,并有可能进一步改进以处理复杂的C语言结构。

文献\cite{he2016experience}介绍了异常检测在现代大规模分布式系统管理中的重要作用。
日志作为记录系统运行信息的工具,在异常检测中被广泛应用。
传统上,开发人员或操作员通常通过关键字搜索和规则匹配手动检查日志。
然而,现代系统的规模和复杂性不断增加,导致日志数量急剧增加,使得手动检查的方法变得不可行。
为了减少人工工作量,提出了许多基于自动化日志分析的异常检测方法。
然而,开发人员可能仍然不知道应该采用哪种异常检测方法,因为这些方法之间缺乏综述和比较。
此外,即使开发人员决定采用一种异常检测方法,重新实现也需要很大的努力。
为了解决这些问题,作者提供了六种最先进的基于日志的异常检测方法的详细综述和评估,
包括三种有监督方法和三种无监督方法,并发布了一个开源工具包,方便复用。

故障诊断日志可以在软件系统出现故障时显著缩短系统恢复时间。
日志自动化工具可以帮助开发人员编写高质量的日志代码。
传统的日志自动化工具设计通过提取语法特征或总结代码模式来定义日志放置规则。
然而,这些方法存在局限性,因为日志放置远不止这些规则,
而是根据软件代码的意图来确定。
为了克服这些限制,文献6\cite{li2020swisslog}设计并实现了SmartLog,
这是一种基于意图感知的日志自动化工具。
为了描述日志语句的意图,该文作者提出了意图描述模型(IDM),
然后探索现有日志的意图,并从等效意图中挖掘日志规则,
并基于6个真实的开源项目进行了实验。
实验结果表明,SmartLog在日志放置准确性方面相比两种最先进的方法分别提高了43\%和16\%。
对于86个旨在添加日志的真实补丁,SmartLog覆盖了其中的57\%,而所有附加日志的开销不足1\%。

文献\cite{zou2014improving}提出了一个集成的故障日志分析平台(UiLog)
,用于收集和管理各种组件日志,为管理员快速定位故障和分析故障原因提供支持。
此系统已经部署在一个实际的云环境中,帮助管理员进行故障排查并找到故障的根本原因。
首先作者提出了一个新的故障日志分类方法,
使用故障关键词矩阵提高了准确性,减少了确定故障类型的时间和手动处理的工作量。
此外,作者改进了现有的日志关联分析方法。结合故障分类结果和时间窗口关联分析,
使用日志的故障类型作为确定时间窗口大小的一个因素,
提高了日志关联分析的准确性和故障根因定位的准确性。
最后,作者介绍了一个综合日志管理系统,帮助管理员快速掌握系统的运行状态,节省故障排查时间。


\section{本文研究内容}
本工具原型首先试图分析程序,得到描述日志记录信息的非确定有穷自动机,并将程序中的控制流信息转换为有穷自动机。该自动机通过状态和转换抽象地记录了程序可能的行为。然后使用自动机识别并分析日志序列,根据相应的转换序列来获得程序运行时刻的信息。

\section{本文结构安排}
本文的主要内容分成从以下六章进行论述。

第一章:绪论。介绍该工具原型设计的意义及目的,阐述了当前相关领域的发展情况,并大体概括了本工具原型的设计内容。

第二章:相关技术介绍,对本文中所应用的相关技术进行介绍描述,包括Clang,libeling库,抽象语法分析树,程序控制流图以及自动机等的相关知识。

第三章:工具原型设计,大致描述了整个工具原型的流程设计,对本工具原型中所采用主要数据结构进行了解释。

第四章:工具原型实现,提供了相应的开发环境安装流程,以及对工具原型每一个功能具体实现做出了详细的解释。

第五章:工具原型测试,针对工具原型功能的可行性验证,以及对工具原型稳定性的测试

第六章:总结与展望,对工具原型的整体功能做出总结,并讨论了工具原型未来的改进与拓展
\section{本章小结}
本章节介绍了现代软件系统中错误定位的重要性和挑战,并分析了现有的错误定位方法和工具,指出了它们在效率和精确度方面的不足,借此,本文提出了一种新的基于 Clang 和自动机的方法,以解析 C语言 程序的日志记录,并以此更有效地定位错误。

