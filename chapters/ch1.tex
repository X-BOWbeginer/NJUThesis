\chapter{绪论}

\section{设计目的及意义}
在大型系统的开发中,日志是分布在函数的不同分支中的,开发人员在测试时需要基于系统的运行日志中的错误信息,来定位错误发生的位置,其中包括了错误发生时的函数调用栈,比如路径敏感分析;跨函数跨文件分析等上下文敏感分析。传统分析方式耗时耗力,需要开发人员对程序本身以及日志插入位置有着深刻的了解,因此提供一种高效、准确的程序流图生成与日志错误定位工具,解决运行日志错误分析过程中的难点,并帮助用户快速定位和定界与特定日志相关的根因,为程序开发和故障排查带来极大的便利和准确性。


\section{相关领域发展现状}
\href{https://users.encs.concordia.ca/~shang/pubs/Zhenhao_ASE20.pdf}{Where Shall We Log? Studying and Suggesting Logging
Locations in Code Blocks} 提供了在不同位置插入日志函数时的效果以及方法论,并用AI进行优化。

\href{https://doc.taixueshu.com/journal/20110163rjxb.html}{程序代码中隐含数据与控制的Petri网建模技术} 研究了程序中的数据与控制流之间的交互以及程序中数据,操作和资源之间的关系,在不运行程序的前提下,生成静态映射,并利用Petri网理论对程序的性质进行分析。

\href{https://github.com/shramos/pyc-cfg}{pyc-cfg} 是一个完成度很高的Python控制流图构建器,适用于几乎所有 Ansi C 编程语言。其原理是根据Clang生成的抽象语法树构建以代码块为单位的程序控制流图,并通过与libclang的python binding来访问它。
\section{设计内容}
本系统首先试图分析程序,得到描述日志记录信息的非确定有穷自动机,然后利用子集构造法,将其转化为确定态有穷自动机,再使用自动机和边上的标记来识别日志序列,最后得到函数的调用栈的信息。

\section{本文结构安排}
本文主要从以下五章进行主要叙述。

第一章:绪论。介绍该系统设计的意义及目的,阐述了当前相关领域的发展情况,并大体概括了本系统的设计内容。

第二章:系统技术介绍,对本文中所应用的相关技术进行介绍描述,包括clang,libeling库,抽象语法分析树,程序控制流图以及自动机进行详尽介绍。

第三章:系统技术介绍

第四章:系统实现

第五章:系统测试

