\chapter{总结与展望}
\section{本文工作}
在现代软件开发和维护过程中,随着系统规模和复杂性的不断增加,
如何高效定位和解决错误变得越来越重要。
传统的错误定位方法往往需要耗费大量时间,尤其在面对复杂的大型系统时,
手动排查错误变得异常艰难。为了应对这一挑战,
本文提出一种通过分析日志记录来提高软件错误定位的效率和准确性的方法,
这个方法具体来说分为以下几个关键步骤:
\begin{enumerate}
    \item 抽象语法树(AST)生成与解析:使用Clang工具生成待分析C语言程序
        的抽象语法树(AST)。
        AST是代码的结构化表示,
        可以帮助作者理解程序的控制流和数据流。
        通过Libclang的Python接口,解析生成的AST。
        这一过程使工具原型能够提取程序的结构信息,
        并自动化地生成程序的NFA。

    \item 从NFA到DFA的转换:基于程序中的日志函数,
        将NFA转换为确定性有限状态自动机(DFA)。
        DFA通过消除NFA中的不确定性,
        使得每个状态在遇到相同输入时都有唯一的后续状态。
        这一转换生成了一个日志链,
        能够描述程序在不同状态下根据日志输出的行为模式。
    \item 日志链的构建与推理:
            通过分析生成的DFA,构建基于代码日志关系的日志链。
            日志链能够展示程序在执行过程中经过的所有状态和状态转移。
            利用日志链对运行时的日志进行推理,
            从而定位程序中的错误位置。通过比对实际日志和预期日志链的差异,
            可以快速发现和定位错误。

    \item 可视化展示:使用NetworkX和Gephi进行可视化展示。
    NetworkX用于在Python中构建和操作图数据结构,
    Gephi则是一个用于探索和可视化大型图的开源平台。
    可视化帮助作者直观地理解程序的控制流和状态转换,
    以及日志链中各个节点和边之间的关系。
\end{enumerate}


\section{未来工作}
尽管本文的方法在提高日志记录的利用效率和定位软件错误方面展现了良好的潜力,
但仍有许多方面可以进一步改进和扩展。未来的研究工作将集中在以下几个方面:
\begin{enumerate}
    \item 目前的方法主要集中在单个模块或组件的日志分析,是一个原理性验证的原型工具。
    未来,作者将扩展到更大规模的系统,涵盖跨模块、跨组件的复杂调用关系。
    \item 未来的工作将致力于优化当前算法和工具的性能,
    使其能够处理更大规模的日志数据和更复杂的控制流。
    \item  除了现有的日志格式,作者计划支持更多种类的日志记录,并挖掘日志中的更多语义信息。
    例如,解析不同类型的异常、错误码和上下文信息。
    \item 当前的可视化展示主要基于静态的DFA图。
    未来,作者将探讨如何动态地展示程序运行过程中的状态变化,
    使开发者能够实时追踪程序执行和状态转移。
\end{enumerate}
\section{本章小结}
最后一章总结了整个研究工作的主要贡献,并展示了其在大型复杂系统中的应用潜力。此外,本文提出了未来工作的发展方向,如进一步优化日志链的生成、提升工具原型的可视化功能以及扩展工具的适用范围。